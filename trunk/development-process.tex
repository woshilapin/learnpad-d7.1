\chapter{Development process}
\label{ch:development-process}

\section{Documentation}
\label{sec:documentation}

There are a few occasions where documentation will be needed such as architectural design,
configuration and deployment, produced source code and complex algorithms.

\subsection{Architecture design}
\label{sec:architecture-design}

% TODO(cjd): Is "set in stone" ok to use in a paper?
According to the Agile Development Methodology, architectural design should not be set in
stone and changes can be made as experience is gained. Therefor architectural design
decisions need not be perfect the first time, better to have something quickly which
satisfies the obvious use cases than to try to project every possible need.

Discussion of architectural design should take place on the mailing list or in logged
chatroome where there is a record of the thoughts and reasoning, as ideas solidify and
consensus forms, they should move to the wiki where they becomes official documentation.

For producing diagrams, a server is available for collaborative work using MagicDraw.
When applicable, MagicDraw can be used for development of ideas on concert with the
mailing list. To minimize the number of places where documentation resides, ideas and
diagrams which have solidified should be exported and placed in the wiki or in the source
repository as appropriate.


\subsection{Configuration and Deployment}
\label{sec:deployment-process}

Since the system will be automatically built and installed on a live test server
(see Section~\ref{live-snapshot}), each component shall be responsible for functioning
with default configuration or none at all.

However, real world implementations will likely look significantly different than the live
snapshot so there is much room for configuration and configuration options require
documentation.
%TODO(cjd): should config documentation be put in a special place where it can be aggregated?

\subsection{Source code}
\label{sec:source-code}

%TODO(cjd): Is this really necessary?
It exists a lot of supporting tools to document code (Doxygen, JavaDoc, PythonDoc, etc.).
No matter what tool is used, each code should be documented at least at the file level: what the
script is for? what the class is for? etc.

\subsection{Complex algorithms}
\label{sec:complex-algorithms}

Some specific parts of source code needs a specific documentation.
Complex algorithms for example, needs a detailed and structured documentation.
Basic source code documentation can be done (see Section~\ref{sec:source-code}) but in these cases,
documentation on the wiki or publication of a paper may happen.

\section{Source management}
\label{sec:source-management}

As already said, all source code will be store in a single repository, using a distributed version
control system like git, mercurial, Subversion, etc.
To ease the collaboration, each service will takes place in one subdirectory of this main
repository. The source will be published on a public server, and also accessible with an web
front-end to ease some of the manipulations (like GitHub, BitBucket, etc.).

\section{Building}
\label{sec:building}

As seen in the Figure~\ref{fig:development-workflow}, a continuous integration will support the
development of the platform. That means that every service should be buildable automatically.
As seen in Source Management (see Section~\ref{sec:source-management}), each service will be a
folder in the source repository. This should allow more freedom in the management of development
activities (tools used, subdirectories, etc.). To ease the global build, each service must provide
a file to automatically build it (for example, each folder will contain a \texttt{build.sh} file).
At the root of the repository, a global builder will parse all of the services to build them.
If specific deployment procedures must be done, they should also appear in this build process.

\section{Testing}
\label{sec:testing}

Testing is an important part for the continuous integration.
Unit tests should be developed as the service level; they will help to check the builds and avoid
regressions as much as possible. Integration tests should allow the continuous integration that
everything is right during a standard deployment with the last versions of all the services.

For unit tests, each service must provide a way to run the tests (for example, by giving a
\texttt{test.sh} file or by allowing a \texttt{test} functionality to the \texttt{build.sh}).
Integration and functional tests are higher levels and should be defined in collaboration.

\section{Bug tracking}
\label{sec:bug-tracking}

Each bug will be tracked with a centralized tool like JIRA, Mantis, GitHub, etc.
The procedure is to report bug each time you encounter problem using services of other partners.
Each partner can also use the tool for reporting its own bugs.
The bug tracking system must be able to notify users by email.
