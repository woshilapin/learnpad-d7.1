\chapter{Integration plan}
\label{ch:integration-plan}

\subsection{Continuous integration}
\label{sec:continuous-integration}

Continuous integration is an automated process by which the current state of the source code is
built into a final product, any automated tests are run and the final product is made available
to developers so that they can test the interaction of their commits in near-real-time. The platform
will be built and deployed on the \emph{target system}, a computer running an installation of
Ubuntu 12.04 LTS \footnote{The stable version of a widely used Linux distribution.}.
Automated builds may also be used as an aid for integrators to do "smoke test" validation on pull
requests before accepting them. In order to protect the working environment of all developers,
it is critical that integrators reject or revert any pull request which causes an incorrect build
of the platform.

\subsection{Live snapshot}
\label{sec:live-snapshot}

Periodically, the most recent build of the \learnpad platform will be automatically published
on a live server running the \emph{target system} so that it can be accessible to all of the
collaborators. This will aid in bug detection and provide a base for discussion of additional
features and the direction of the project.

\subsection{Release Process}
\label{sec:release-process}

As described in the Global Roadmap (see Section~\ref{sec:global-roadmap}), releases will happen
every 6 Development Milestones. In the final 2 week development cycle before the release, a stable
branch of the source code will be created and partners will be asked to test the features for
which they are responsible. The live snapshot will be available for this. During these two weeks,
only minor bug-fix patches will be accepted to the stable branch and only if Integrators judge them
to be low risk to the release. At the end of the 2 week development cycle, a final version of the
release will be created and made available for download by the partners and stakeholders.
